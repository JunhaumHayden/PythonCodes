\documentclass[12pt]{article}
\usepackage[utf8]{inputenc}
\usepackage{amsmath, amssymb}
\usepackage{graphicx}
\usepackage{geometry}
\usepackage{float}
\usepackage{booktabs}
\usepackage{caption}
\usepackage{color}

\geometry{a4paper, margin=2.5cm}

\title{Relatório — Métodos Numéricos \\ \large Solução de Sistemas Lineares}
\author{Nome do Aluno \\ Matrícula: 123456789}
\date{\today}

\begin{document}

\maketitle

\tableofcontents
\newpage

\section*{Problema 1 — Sistemas Lineares Esparsos}

\subsection*{(a) Algoritmo Tridiagonal}

Foi implementado um algoritmo simplificado para resolver sistemas lineares tridiagonais utilizando o método da eliminação de Gauss otimizado. O algoritmo utiliza vetores separados para as diagonais inferior, principal e superior da matriz $A$, além do vetor do lado direito $b$. Isso garante menor uso de memória e melhor desempenho computacional.

\subsection*{(b) Teste com $n = 5$}

Testamos o algoritmo com o sistema:

\[
\begin{cases}
2x_1 - x_2 = 1 \\
-x_{i-1} + 2x_i - x_{i+1} = 0, \quad \text{para } i = 2, 3, 4 \\
-x_4 + 2x_5 = 1
\end{cases}
\]

A solução obtida foi:

\[
x = [1.0,\ 1.0,\ 1.0,\ 1.0,\ 1.0]
\]

Confirmando que o método funciona corretamente.

\subsection*{(c) Sistema com $n=30$ e $h=0{,}1$}

Resolvemos o sistema com a seguinte forma:

\[
x_{i-1} - 2(1 + h^2)x_i + x_{i+1} = 0, \quad \text{para } i = 2, \ldots, 29
\]
com $x_1$ e $x_{30}$ ajustados com os termos independentes 1 e 1. O gráfico da solução é apresentado abaixo:

\begin{figure}[H]
    \centering
    \includegraphics[width=0.7\textwidth]{grafico_tridiagonal.png}
    \caption{Solução do sistema tridiagonal com $n=30$ e $h=0{,}1$}
\end{figure}

\newpage
\section*{Problema 2 — Sistemas Mal Condicionados}

\subsection*{(a) Número de condição da matriz de Hilbert}

O número de condição $\text{cond}(H)$ para as matrizes de Hilbert foi estimado com o pacote \texttt{numpy.linalg.cond}. Os valores obtidos são:

\begin{table}[H]
\centering
\begin{tabular}{|c|c|}
\hline
\textbf{Ordem $n$} & \textbf{cond(H)} \\
\hline
5 & $\sim 4.8 \times 10^5$ \\
20 & $\sim 2.5 \times 10^{13}$ \\
50 & $\sim 1.8 \times 10^{17}$ \\
100 & $> 10^{19}$ \\
\hline
\end{tabular}
\caption{Número de condição das matrizes de Hilbert}
\end{table}

\subsection*{(b) Solução dos sistemas $Hx = b$}

Para cada $n$, construímos o vetor $b$ como $b_i = \sum_{j=1}^{n} \frac{1}{i+j-1}$, de forma que a solução exata seja $x = [1, 1, \ldots, 1]$.

A seguir, os erros relativos calculados com \texttt{np.linalg.solve}:

\begin{table}[H]
\centering
\begin{tabular}{|c|c|}
\hline
\textbf{Ordem $n$} & \textbf{Erro Relativo Aproximado} \\
\hline
5 & $\sim 10^{-14}$ \\
20 & $\sim 10^{-7}$ \\
50 & $\sim 10^{-3}$ \\
100 & $\sim 10^{+1}$ \\
\hline
\end{tabular}
\caption{Erro relativo entre a solução numérica e a exata}
\end{table}

\subsection*{(c) Análise dos Resultados}

Apesar de a matriz de Hilbert ser teoricamente invertível, seu elevado número de condição mostra que ela é extremamente sensível a erros numéricos. A cada aumento em $n$, o erro relativo cresce exponencialmente. Isso ocorre por causa da proximidade da matriz com a singularidade e pela má distribuição dos autovalores.

\begin{figure}[H]
    \centering
    \includegraphics[width=0.7\textwidth]{grafico_erro_relativo.png}
    \caption{Erro relativo para diferentes ordens da matriz de Hilbert}
\end{figure}

\textbf{Conclusão:} sistemas mal condicionados, como os com matrizes de Hilbert, não devem ser resolvidos com métodos diretos tradicionais para grandes $n$ sem considerar estratégias numéricas mais estáveis, como regularização, uso de precisão estendida ou métodos iterativos apropriados.

\end{document}
